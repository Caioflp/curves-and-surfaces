\section{Geodésicas}

\begin{defn}
    Uma curva parametrizada diferenciável em uma superfície regular \( S \), \( \gamma : I \subseteq \R \to S \), é dita uma \emph{geodésica parametrizada} se seu vetor aceleração é sempre perpendicular ao espaço tangente da superfície, ou seja, se para todo \( p = \gamma ( t ) \in \gamma ( I ) \) temos \( \gamma'' ( t ) \in \left\{ T_{ \gamma ( t ) } S \right\}^{ \perp } \).
    Analogamente, uma curva regular \( C \subset S \) é dita uma \emph{geodésica} de \( S \) se, para todo ponto \( p \in C \), existe uma parametrização local \( \gamma : I \subseteq \R \to C \), de \( C \) em \( p \), tal que \( \gamma \) é uma geodésica parametrizada.

\comment{
    norma do vetor velocidade da geodesica é constante
    caracterização com produto interno da aceleracao e a normal de gauss *
    equacoes diferenciais
    existencia e unicidade *
    invariancia por isometrias locais
    homogeneidade
}
    
\end{defn}

