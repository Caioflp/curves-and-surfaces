\section{Isometrias}

Quando estamos trabalhando com aplicações de \( \R^{ n } \) em \( \R^{ m } \), isometrias são aquelas que preservam o comprimento de vetores.
Ao mudar o ambiente de trabalho para superfícies, a definição de isometria faz referência à derivada da aplição e, consequentemente, aos espaços tangentes a essas superfícies.
Mais explicitamente, temos:
\begin{defn}[Isometria]
    Dadas superfícies regulares \( S_{ 1 } \) e \( S_{ 2 } \), dizemos que uma aplicação \( f : S_{ 1 } \to S_{ 2 } \) é uma \emph{isometria} se é um difeomorfismo e, para todo \( p \in S_{ 1 } \) e todos \( w, v \in T_{ p } S_{ 1 } \) temos
    \begin{equation}
        \dotprod{ df_{ p } w, df_{ p } v } = \dotprod{ w, v }
        \label{eq: isometria}
    .\end{equation}
    Ainda, dizemos que \( f : S_{ 1 } \to S_{ 2 } \) é uma \emph{isometria local} se, para todo \( p \in S_{ 1 } \) existem abertos \( V_{ 1 } \subseteq S_{ 1 } \) e \( V_{ 2 } \subseteq V_{ 2 } \), tais que \( p \in V_{ 1 }, f ( p ) \in V_{ 2 } \) e \( f \mid_{ V_{ 1 } } : V_{ 1 } \to V_{ 2 } \) é uma isometria.

Finalmente, duas superfícies regulares são ditas \emph{isométricas} se existe uma isometria entre elas.
\end{defn}

Agora, alguns comentários.
Observe que, se \( f \) é isometria local e difeomorfismo, então é isometria.
Também perceba que, pela \emph{identidade de polarização}:
\begin{equation*}
    \dotprod{ w, v } = \frac{ \norm{ w + v }^2 - \norm{ w - v }^2 }{ 4 }
\end{equation*}
a condição (\ref{eq: isometria}) é equivalente a termos
\begin{equation}
    \norm{ df_{ p }w } = \norm{ w }
    \label{eq: isometria2}
\end{equation}
para todo \( w \in T_{ p } S_{ 1 } \).

Observe, ainda, que se \( f : S_{ 1 } \to S_{ 2 } \) cumpre (\ref{eq: isometria2}) (ou, equivalentemente, (\ref{eq: isometria}), pelo que acabamos de mostrar), então, para todo \( p \in S_{ 1 } \), temos que \( df_{ p } \) é um isomorfismo.
De fato, a injetividade vem do fato de que se \( df_{ p }w = df_{ p }v \), então
\begin{equation*}
    \norm{ w - v } = \norm{ df_{ p } ( w - v ) } = \norm{ df_{ p }w - df_{ p }v } = 0
,\end{equation*}
de modo que \( w = v \).
Agora, como \( 2 = \dim T_{ f ( p ) } S_{ 2 } \geq \dim df_{ p } ( T_{ p } S_{ 1 } ) = 2 \), necessariamente temos \( T_{ f ( p ) } S_{ 2 } = df_{ p } ( T_{ p } S_{ 1 } ) \), ou seja, \( df_{ p } \) é sobrejetiva.
Dessa forma, pelo Teorema da Função Inversa, para cada \( p \in S_{ 1 } \) existem vizinhanças \( V_{ 1 } \ni p \) e \( V_{ 2 } \ni f ( p ) \) em \( S_{ 1 } \) e \( S_{ 2 } \), respectivamente, tais que \( f\mid_{ V_{ 1 } } : V_{ 1 } \to V_{ 2 } \) é difeomorfismo.
Ou seja, \( f \) é isometria local.

Com isso, provamos a seguinte proposição, que caracteriza as isometrias locais.
\begin{prop}
    Uma aplicação entre superfícies regulares é uma isometria local se, e somente se, preserva a primeira forma fundamental.
\end{prop}

Dado esse resultado, podemos esperar que isometrias locais preservem, de uma certa forma, os coeficientes da primeira forma fundamental.
Isso de fato ocorre, como podemos ver na seguinte proposição:

\begin{prop}
    Sejam \( S_{ 1 } \) e \( S_{ 2 } \) superfícies regulares, \( f : S_{ 1 } \to S_{ 2 } \) uma isometria local e \( X : U \subseteq \R^{ 2 } \to S_{ 1 } \) uma parametrização local de \( S_{ 1 } \), cuja imagem \( X ( U ) \defeq V \) é um aberto de \( S_{ 1 } \) tal que \( f \mid_{ V } : V \to f ( V ) \subseteq S_{ 2 } \) é uma isometria.
    Sob essas hipóteses, \( Y = f \circ X : U \to S^{ 2 } \) é uma parametrização local de \( S_{ 2 } \), cujos coeficientes da primeira forma fundamental coincidem com aqueles da parametrização \( X \).
    \label{conserva primeira forma}
\end{prop}

\begin{proof}
    Como a composição de difeomorfismos é um difeomorfismo e tanto \( f \mid_{ V } \) quanto \( X \) são difeomorfismos, temos que \( Y \) é uma parametrização local de \( S_{ 2 } \).
    Além disso, como a restrição de \( f \) a \( V \) é uma isometria, temos, para todo \( ( u, v ) \in U \):
    \begin{align*}
        \dotprod{ Y_{ u } ( u, v ), Y_{ u } ( u, v ) }
        &= \dotprod{
            ( f \circ X )_{ u } ( u, v ),
            ( f \circ X )_{ u } ( u, v )
        } \\
        &= \dotprod{
            df_{ X ( u, v ) } X_{ u } ( u, v ),
            df_{ X ( u, v ) } X_{ u } ( u, v )
        } \\
        &= \dotprod{ 
            X_{ u } ( u, v ), X_{ u } ( u, v )
        }
    .\end{align*}
    Ou seja, o coeficiente \( E \) da primeira forma fundamental é preservado.
    Analogamente demonstra-se que
    \begin{equation*}
        \dotprod{ Y_{ u }, Y_{ v } } = \dotprod{ X_{ u }, X_{ v } }
        \ \text{ e } \
        \dotprod{ Y_{ v }, Y_{ v } } = \dotprod{ X_{ v }, X_{ v } }
    .\end{equation*}
    Com isso, todos os coeficientes da primeira forma fundamental são preservados, como queríamos demonstrar.
\end{proof}
