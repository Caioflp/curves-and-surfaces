\section{Introdução}

Neste trabalho, vamos definir o conceito de geodésica e estudar suas propriedades mais elementares, culminando com a demonstração de sua existência e unicidade.
A exposição foi integralmente baseada no capítulo 4 de \cite{ronaldo}.
Apesar de não chegarmos a abordar esta parte, é importante enfatizar que o interesse sobre geodésicas vem do fato de que, \emph{localmente}, elas minimizam o comprimento de caminhos entre pontos na superfície.
Isso poderá ser visualizado quando mostrarmos alguns exemplos de geodésicas.
Antes de tudo, porém, vamos apenas revisar um pouco da notação, bem como um teorema, que serão utilizados.

Denotamos o produto vetorial entre dois vetores \( v, w \in \R^{ 3 } \) por \( v \times w \).
A esfera unitária em \( \R^{ 3 } \), isto é, o conjunto \( ( x, y, z ) \in \R^{ 3 } : x^2 + y^2 + z^2 = 1 \) será denotada por \( \mathbb{S}^2 \).
Quando escrevermos, por exemplo, \( t \mapsto t^2 \), estaremos fazendo referência à \emph{função} que mapeia um número \( t \) ao número \( t^2 \).
Além disso, dada uma função \( f \), a notação \( f \equiv 0 \) significa que \( f \) é \emph{identicamente nula}, ou seja, \( f ( x ) = 0 \) para todo \( x \) no domínio de \( f \).
Por fim, relembramos o enunciado do teorema que garante a existência e unicidade das soluções de uma equação diferencial ordinária:
\begin{teo*}
    Seja \( D \subseteq \R \times \R^{ n } \) um retângulo fechado com \( ( t_{ 0 }, u_{ 0 } ) \in D \).
    Seja \( f : D \to \R^{ n } \) uma função contínua em \( t \) e Lipschitz contínua em \( y \).
    Então, existe \( \varepsilon > 0 \) tal que a equação
    \begin{equation*}
        y' ( t ) = f ( t, y ( t ) )
    ,\end{equation*}
    junto com a restrição \( y ( t_{ 0 } ) = y_{ 0 } \) tem uma solução única no intervalo \( ( t_{ 0 } - \varepsilon, t_{ 0 } + \varepsilon ) \).
\end{teo*}


%% \( \times \)
%% \( \mathbb{S}^{ n } \)
%% EG - F^2 > 0
%% \( x \mapsto f ( x ) \)
%% \( f \equiv 0 \)
